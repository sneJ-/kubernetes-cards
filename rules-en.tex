\documentclass{article}
\usepackage{geometry}
\usepackage{enumitem}
\geometry{a4paper, margin=1in}

\title{KubeMastery \\ \large Kubernetes Card Game Rules}
\author{}
\date{}

\begin{document}

\maketitle

\section*{Objective}
KubeMastery is a card game designed to help players learn and familiarize themselves with Kubernetes concepts. Players will guess, describe, and strategically use Kubernetes resource cards to complete challenges and earn points.

\section*{Setup}
\begin{itemize}
    \item \textbf{Players:} 2-6
    \item \textbf{Deck:} 32 Kubernetes resource cards
    \item \textbf{Categories:} Workload, Network, RBAC, Configuration, Storage, Event, Misc
    \item \textbf{Markers:} Namespaced or Global resources
    \item \textbf{Tokens/Points:} Used to track progress
\end{itemize}

\section*{Game Loop}
The game is divided into three phases: Guess and Describe, KubeChallenges, and KubeSprint.

\subsection*{Phase 1: Guess and Describe (Memory Phase)}
\begin{enumerate}[label=\arabic*.]
    \item \textbf{Draw and Relay:} The player to the left of the active player (the one who will describe the resource) draws a card \textbf{without showing} it to anyone. They \textbf{only reveal the title} of the card to the player on their right (the active player).
    
    \item \textbf{Describe the Resource:} The active player must describe the resource from memory, covering:
    \begin{itemize}
        \item \textbf{Category} (Workload, Network, etc.)
        \item \textbf{Whether it is Namespaced or Global}
        \item \textbf{Key function} or \textbf{usage} of the resource
    \end{itemize}
    The player does this \textbf{without seeing the card}, relying entirely on their knowledge.
    
    \item \textbf{Reveal and Score:} Once the active player has finished their description, the card is revealed to everyone.
    \begin{itemize}
        \item \textbf{1 point} for each correct detail (Category, Namespaced/Global, Key function).
        \item If the player misses any details, the other players have the opportunity to \textbf{steal the point}:
        \begin{itemize}
            \item Players who know the correct detail can "buzz in" by saying "Buzz!" or raising their hand quickly.
            \item The first player to buzz in or raise their hand gets the first opportunity to steal.
            \item If multiple players buzz in at the same time, the chance to steal rotates clockwise starting from the player to the left of the active player.
            \item Each player has only \textbf{one chance} to provide the correct detail. If they answer correctly, they steal the point. If incorrect, the next player in the rotation gets a chance.
        \end{itemize}
        \item If no player can correctly identify the missing detail, the point is forfeited.
    \end{itemize}
    
    \item \textbf{Group Discussion:} After scoring, the group briefly discusses the resource to ensure everyone understands it. The card is then placed face-up on the table so everyone can see and review it.
\end{enumerate}

\subsection*{Phase 2: KubeChallenges (Strategy Phase)}
\begin{enumerate}[label=\arabic*.]
    \item \textbf{Setup a Kubernetes Challenge:} Each round, one player becomes the “KubeMaster” and sets up a Kubernetes challenge based on a real-world scenario (e.g., "How would you ensure high availability for a web application?").
    
    \item \textbf{Select and Play Cards:} Players choose a card from their hand (or draw from the deck) that they think best addresses the challenge. Each player explains how their selected resource solves or contributes to the solution for the scenario.
    
    \item \textbf{KubeMaster's Judgement:} The KubeMaster listens to all explanations and then awards \textbf{3 points} to the player with the best solution. \textbf{1 point} is awarded to all other players who provided valid, but less optimal, solutions.
    
    \item \textbf{Rotation:} The role of KubeMaster rotates to the next player, and a new challenge is set.
\end{enumerate}

\subsection*{Phase 3: KubeSprint (Speed Phase)}
\begin{enumerate}[label=\arabic*.]
    \item \textbf{Rapid Fire Questions:} Players go through a rapid-fire round where the current KubeMaster asks quick questions related to the cards, such as:
    \begin{itemize}
        \item "Name a resource that is always Namespaced."
        \item "Which resource is used for managing secrets?"
    \end{itemize}
    
    \item \textbf{First to Answer:} The first player to correctly answer the question gets \textbf{2 points}. If no one answers correctly, the KubeMaster can give a hint, reducing the points to \textbf{1} for the correct answer.
    
    \item \textbf{End of Sprint:} After a set number of questions (e.g., 5-10), this phase ends, and points are tallied.
\end{enumerate}

\section*{Winning the Game}
The game can be played over several rounds, with the player accumulating the most points by the end winning the game. Alternatively, set a target score, and the first player to reach it wins.

\section*{Benefits}
\begin{itemize}
    \item \textbf{Fair Play:} The player describing the resource can't see the card, which ensures they rely on their memory.
    \item \textbf{Group Learning:} The discussion and review of each card after it’s revealed help reinforce knowledge.
    \item \textbf{Engagement:} The game remains interactive, with each phase providing a different way to engage with the Kubernetes concepts.
\end{itemize}

\end{document}
